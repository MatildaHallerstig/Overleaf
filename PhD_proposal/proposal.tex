\documentclass[a4paper,10pt]{article}
\usepackage[utf8]{inputenc}

%opening
\title{The performance of a regional numerical weather prediction model in the maritime Arctic}
\author{}

\begin{document}
\raggedright

\maketitle

% make to table:
% \begin{abstract}
Student: Matilda Hallerstig

Supervisor: Erik Kolstad, NORCE 

Co-supervisor: Harald Sodemann, UiB

Co-supervisor: Marius Jonassen, UNIS
% \end{abstract}

\section{Introduction}
% (Faglig bakgrunn ("state-of-the-art") med sentrale referanser. Her inngår kandidatens kunnskapsstatus, samt hvordan det planlagte prosjektet vil tilføre ny kunnskap til det angitte forskningsfeltet.)

% Why this project matters/the issues

% Briefly what arome is

% A review of relevant work that has been done (take references from the alertness project description)

% Weather forecasting challenges in the Arctic
% Rapid climatic and environmental changes, and an increasing human presence in the region, have all triggered an immediate need for both applied and basic research advances to improve Arctic weather prediction. Marine icing, fog, polar lows, strong winds and high waves are major hazards to marine operations and industrial development.

There is a growing interest in Arctic operations due to socio-economic opportunities. Fuelled by increased accessibility due to recent sea ice retreat, activities related to exploration, tourism, transportation and scientific research are expected to attract economic investments exceeding \$100bn (Lloyd’s 2012) over the coming decade. However, Arctic weather can be a hazard to high-latitude activities and infrastructure, such as shipping, fishery, gas and oil exploration and exploitation, land transport and aviation. There is an urgent need for research towards reliable and accurate polar weather prediction capabilities. 

NWP models generally show lower forecast capability at high latitudes compared to other regions (Jung et al. 2016). This is partly because Arctic weather systems pose challenges different to those at mid-latitudes for which most of these models are developed, and partly due to the scarcity of in-situ observations. Forecasting high-impact weather events in the Arctic has proven to be especially challenging. Repeated severe forecast misses, aggravated by fast climatic change inducing unusual weather, have had dramatic consequences for local communities. Such high impact weather events include intense and rapidly developing mesoscale cyclones known as polar lows embedded in large cold-air outbreaks characterised by convective processes (Kolstad 2017), icing conditions from sea spray during winter (Samuelsen et al. 2017), episodes of persistent fog during summer and aviation icing (Gultepe et al. 2015), and avalanche and landslide risks after heavy precipitation. % In the Arctic, the societal value of fundamental weather research is strongly conditioned upon the ability to provide forecasts and warnings that user groups can incorporate in their decision-making processes.

A major challenge in the Arctic is that small-scale processes and variability are particularly relevant for the accuracy of a forecast (WMO-PPP 2013). While large-scale circulation patterns may be reasonably predictable several days in advance, mesoscale weather (e.g. polar lows) is strongly influenced by parameterized, sub-grid scale processes, such as surface fluxes, radiation, convection and cloud microphysics, and their interaction, which in many cases are highly uncertain in polar regions (Vihma et al. 2014) and not always well represented in NWP models.

 The importance of kilometre-scale grid spacing for the forecast quality of polar lows has repeatedly been highlighted (e.g. Kristiansen et al. 2011), and is explored in several research projects (e.g. EU-project APPLICATE). Yet even at 2.5 km grid spacing, moist convection, critical for representing PLs (Kolstad et al. 2016), is only partly resolved. Gradual increase of the open water exposure to CAOs leads to more frequent extreme convective events with the heat fluxes exceeding 500 W m -2 (Smedsrud et al. 2013). The large fluxes drive strong self-organized cellular convection, responsible for hail, snow and gale force wind gusts – dangerous, but potentially predictable phenomena, given appropriate parameterisations (Feingold et al. 2010). 

 % repeats what was already said
 % The difficulties for NWP in the Arctic are exacerbated by a sparse conventional observation network, especially over the ocean and sea-ice, which is both error-prone and of limited representativeness (Casati et al. 2017). Observations play a cross-cutting role in the context of NWP in that they are used for understanding and parameterising physical processes, for model verification, and for model initialisation. For verification and process study purposes over the Arctic, past campaigns (e.g. IPY-THORPEX; Kristjansson et al. 2011) have not yet been fully utilized, and several YOPP campaigns are planned. To this end, both development and comprehensive testing of new tailored verification techniques are needed, constituting a key challenge in polar verification (WMO-PPP 2013).
 
% (...)

% The importance of kilometre-scale grid spacing for the forecast quality of polar lows has repeatedly been highlighted (e.g. Kristiansen et al. 2011), and is explored in several research projects (e.g. EU-project APPLICATE). Yet even at 2.5 km grid spacing, moist convection, critical for representing PLs (Kolstad et al. 2016), is only partly resolved. Gradual increase of the open water exposure to CAOs leads to more frequent extreme convective events with the heat fluxes exceeding 500 W m -2 (Smedsrud et al. 2013). The large fluxes drive strong self-organized cellular convection, responsible for hail, snow and gale force wind gusts – dangerous, but potentially predictable phenomena, given appropriate parameterisations (Feingold et al. 2010). Current NWP models act on a per-column basis, so the sensitivity of Arctic weather forecasts to organising convection is as yet unknown. The presence of sea ice is a further complicating factor, which introduces a long-memory component that causes substantial variability of surface conditions (Valkonen et al. 2008). Moreover, the fine-scale structure of leads and polynyas mean that sea ice is difficult to accurately observe and therefore represent in an NWP model.



\section{Aims and objectives}
% ( Faglig problemstilling, metodikk og prosjektmål med tentativ tidsplan. Redegjørelse av hva som forventes å oppnås. Helst etterprøvbare hypoteser.)

% Description of Alertness project and work packages, with special focus on WP1

% Description of my focus and Scientific questions

In response to the urgent need for Arctic weather prediction, a convection-permitting mesoscale model for the Arctic has recently been introduced into service by MET Norway (Müller et al. 2017a). AROME Arctic, an operational short-range, convection-permitting prediction system dedicated to the European Arctic, issues forecasts four times per day with a lead time of 66 hours, at a horizontal grid spacing of 2.5 km. % The important processes of sea ice and interactions across the air–ice–ocean interface are represented by a one-dimensional sea ice model (SICE). 
% Something about what the alertness is here, and description of work packages.
ALERTNESS takes an innovative and comprehensive approach to address the growing need for accurate and reliable weather predictions in the Arctic, especially in relation to high-impact weather. Rapid climatic and environmental changes, and an increasing human presence in the region, have all triggered an immediate need for both applied and basic research advances to improve Arctic weather prediction. Marine icing, fog, polar lows, strong winds and high waves are major hazards to marine operations and industrial development. 

We will take advantage of several unique opportunities arising during the Year of Polar Prediction (YOPP). Our approach is to develop new methods to tackle long-standing issues in atmospheric models in polar environments, and to continually evaluate these methods against data from YOPP observations. We will also enable more comprehensive use of valuable observations from past field campaigns. ALERTNESS will
explore new ways to diagnose uncertainties evolving from representations of small-scale processes, and generate substantial gains in probabilistic forecasting for the Arctic.

An important aspect of ALERTNESS is that academic researchers collaborate directly with operational forecasting centres. Our advances will provide guidance for long-term model improvement by the larger Numerical Weather Prediction (NWP) community, and ensure sustained benefits for the scientific and wider communities. All of our research data and publications will be made available for open access. ALERTNESS will work towards enabling stakeholders in the region to make better informed risk-based decisions. Our work will be guided by recommendations from international strategy documents, invoke the expertise of international partners, and be closely coordinated with several related national and international projects.

ALERTNESS will tackle the Arctic verification problem by using as many (routine and campaign) observation and model data sets as possible for in-depth evaluation of AROME Arctic, including comparison with other YOPP core models. To this end, a reference database of well-observed HIW events will provide a baseline for model developments in WPs 2-4 (see Fig. 2). The database will include episodes from the YOPP special observing periods (SOPs), the YOPP-endorsed Iceland Greenland Seas Project (IGP) aircraft campaign in
March 2018, and other historical campaigns (e.g. IPY-THORPEX). In IGP the project will fund a much needed met buoy co-located with an existing subsurface mooring, set up and run dedicated model simulations and perform model analyses. WP1 will design, develop, use and distribute deterministic and probabilistic verification methods (WPs 2-4), metrics (such as the MET-Norway PL tracking tool; Kristiansen et al. 2011), and diagnostics towards the specific requirements of the polar environment. For mutual benefit between model developers, operational forecasters, related projects (EU projects Blue-Action and APPLICATE) and end-users, WP1 will, in collaboration with WP5, the existing user and stakeholder mechanisms for exchange of requirements, opportunities and experience (see Fig. 3). Special attention will be given to aviation, persistent summertime fog, and maritime icing due to high wind and sea spray. This new set of verification measures appropriate for Arctic weather forecasts will be used throughout the project in tandem with standard measures to monitor progress, including user-relevant parameters. We will follow key research foci in polar verification discussed by Casati et al. (2017): account for observational uncertainty (e.g. Mittermaier 2014; Wolff et al. 2015), enhance synergies between verification
and data assimilation by exploring the use of model analyses for verification (e.g. Randriamampianina et al. 2011; Lemieux et al. 2016), and include spatial and probabilistic verification methods available in the HIRLAM-Aladin R Package (HARP). Events with large forecast misses will be identified and investigated in qualitative case studies to better understand the origins of the forecast errors. Typically, this will happen during YOPP were MET Norway will be one of the centres providing operational support. To enable robust conclusions, these events will be compared with a wider set of (historical) cases.


\subsection{Work plan}
%\begin{center}
 \begin{tabular}{ p{5cm}  c  c  c  c  c  c } 
                                                                                      & 2018 & \multicolumn{2}{c}{2019} & \multicolumn{2}{c}{2020} & 2021\\
 
                                                                                      & Fall & Spring & Fall & Spring & Fall & Spring\\ [0.5ex] 
 \hline
 \hline
 Establish a reference database with test-cases                            & x    &          &       &          &      &\\ [0.5ex]
 \hline
Develop metrics and diagnostics appropriate for the maritime Arctic & x    & x       & x     &          &      &\\[0.5ex]
 \hline
Evaluate model performance during high-impact weather events      &       & x       & x     & x       &      &\\[0.5ex]
 \hline
Analyse the forecast skill of existing and enhanced model              &        &         &       & x       & x    & x\\
 \hline \\ [0.5ex]
\textbf{Courses}                                                                 &       &          &       &         &       &\\ %[1.5ex]% [1ex] 
 \hline
 \hline
E-science tools for climate research (CHESS) 5p                         & x     &          &       &         &       &\\ [0.5ex]% [1ex] 
 \hline
Applied statistics (STAT200) 10p                                            &      & x         &       &         &       &\\ [0.5ex]% [1ex] 
  \hline
 Science communication: Creating scientific illustrations (CHESS)    &      & x         &       &         &       &\\ [0.5ex]% [1ex] 
 \hline
 Predictability and ensemble forecast systems (ECMWF) 2p           &       & x         &      &         &       &\\[0.5ex]% [1ex] 
 \hline
 Microwave satellite remote sensing (GEOF345) 5p                      &       &            &      & x      &       &\\[0.5ex] % [1ex] 
 \hline
 Theory of science and ethics (MNF990) 5p                               &       &            & x    &        &       &\\ [0.5ex]% [1ex] 
 \hline
 \end{tabular}
% \end{center}

\subsection{Publication plan}
 \begin{tabular}{ l  l  l  l } 

            & Topic & Co-authors & Submit \\ [0.5ex] 
 \hline
 \hline
 Paper 1 & Polar lows in ECMWF and Arome Arctic    & Erik, Linus                         & \\ [0.5ex]
 \hline
 Paper 2 & The most beautiful polar low                   & Erik, Eirik                          & \\[0.5ex]
 \hline
 Paper 3 &  Narve                                               & Erik, Eirik                          & \\[0.5ex]
 \hline
  Paper 4 & Evaluation of model improvements (SBL?) & Marvin, Erik, Harald, Marius? & \\[0.5ex]
 \hline
  Paper 5 &  Summertime fog                                 & Erik, Marvin?, Harald?          & \\[0.5ex]
 \hline
  Paper 6 & Aviation icing                                      & Erik, Marius?                     & \\[0.5ex]
 \hline
 \end{tabular}
\section{Methods}

(Datagrunnlag, metoder og statistikk.)

What will be done
Case studies, sensitivity tests

Description of the data
Arome, EC, remote sensing (specially IR and ASCAT), YOPP and IPY field campaigns

Tools (Lustre, Elvis(MET IT-infrastructure))

\section{Cooperation with external institutions}
 (Oversikt over samarbeidspartnere, utenlandsopphold, møtedeltakelse, etc.)

Official collaborators (MET, UNIS in WP1, also mention collaborators in other workpackages?), mention stay in Svalbard, Oslo, Tromsø. Mention alertness meetings, conference in Helsinki. What about EGU and Arctic frontiers? Mention ECMWF, CHESS, Bjerknes and UiB. Courses in Andenes, Reading.

\section{Fundings}

The PhD is fully funded through the ALERTNESS project.
\section{References}

\end{document}
